Traditional signature-based malware detection increasingly fails against evolving variants. This project explores an alternative approach by converting executable files into visual representations and applying convolutional neural networks for binary classification. We present a Python-based pipeline that performs dual-modality analysis: (1)~digram visualization, which captures byte-pair frequency and positional statistics from binary sequences via a 256×256 tensor, and (2)~opcode fingerprinting, which maps x86-64 instruction patterns to a 1024×1024 locality-sensitive hash density map. Generated images are normalized and fed to a three-block CNN with batch normalization, dropout regularization, and aggressive data augmentation (horizontal/vertical flips, rotations, zoom, contrast, translation).

Evaluation on a small, balanced dataset reveals mixed results. The digram model achieves ~75\% validation accuracy but exhibits a marked generalization gap and threshold-sensitive precision/recall trade-offs. The opcode model similarly captures instruction-level structure yet remains unstable under the small validation regime. Confusion matrices and loss curves indicate meaningful learned patterns, but conclusions are statistically fragile, with each misclassification significantly impacting reported metrics on sets of only 12 validation samples.

We conclude that CNN-based visual malware detection, while theoretically promising, requires substantially larger and more diverse labeled datasets to achieve reliable deployment. As an intermediate finding, classical statistical analysis of image similarity clustering, distributional comparisons, and heuristic-based classification emerges as a more dependable exploitation of these visualizations pending adequate data accumulation.
