\section{Abstract}

This project investigates a novel approach to malware detection using binary visualization and convolutional neural networks (CNNs). Traditional signature-based malware detection methods face increasing challenges as adversaries develop new variants at scale. We propose a technique that transforms executable files into visual representations by analyzing their binary sequences and opcodes, converting low-level program data into images suitable for deep learning analysis. A Python-based pipeline processes executable files from an input directory, performing statistical visualization of binary data alongside opcode analysis for executable files. These generated images are subsequently used to train multiple CNN models capable of distinguishing between benign and malicious software. Our approach leverages image classification techniques to detect anomalies in binary behavior patterns, providing an alternative perspective to traditional static and dynamic analysis methods. Experimental results demonstrate that this method achieves promising performance in malware detection, validating the effectiveness of binary visualization as a feature extraction strategy. The findings suggest that deep learning-based visual analysis can be a valid and efficient complement to existing malware detection systems, particularly for identifying novel or obfuscated threats.


